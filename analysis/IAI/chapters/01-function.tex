\chapter{函数}\label{ch 1}

\begin{note}
    在分析中,a,b,c等代表常量,x,y,z等代表变量。确定的量(常量)都只可以是一个数,而变量包含每一个确定的量。指定变量为某个确定的值,它就成了常量。
    变量包容着正数和负数、整数和分数、有理数和超越数、零和虚数等一切数。
    
    变量的函数是变量、常量和数用某种方式联合在一起的解析表达式。只含一个变量$z$,其余均为常量的表达式为$z$的函数,
    \begin{equation*}
        a + 3z, az - 4z^2, az+b \sqrt{a^2 - z^2}, z^2
    \end{equation*}
    
    变量的函数本身也是一个变量。变量可以取虚数值,因为函数也可以去任何值。例如$\sqrt{9 - z^2}$。存在样子像函数的常量,
    \begin{equation*}
        z^0, 1^z, \frac{a^2 - az}{a - z}
    \end{equation*}
    
    函数之间的基本区别是变量和常量的联合方式,联合方式决定于运算。运算包括代数运算,即加、减、乘、除、乘方、开方和解方程;超越运算,即指数运算,对数运算以及积分学提供的运算。下面对包含多于一种运算的表达式加以分类。函数分为\textbf{代数函数}和\textbf{超越函数}。代数函数只含有代数运算,超越函数含有超越运算。代数函数常常不能显式表出。例如$z$的函数$Z$,
    \begin{equation*}
        Z^5 = az^2Z^3 - bz^4Z^2 + cz^3Z - 1
    \end{equation*}
    
    代数函数又分为有理函数和无理函数。有理函数的变量不受根号作用,无理函数反之。有理函数仅含加、减、乘、除和整数次乘方运算,
    \begin{equation*}
        a + z, a - z, az, \frac{a^2 + z^2}{a + z}, az^3 - bz^5
    \end{equation*}
    
    而无理函数形如,
    \begin{equation*}
        a + \sqrt{a^2 - z^2}, (a - 2z + z^2)^{\frac{1}{3}}, \frac{a^2 - z\sqrt{a^2 + z^2}}{a + z}
    \end{equation*}
    
    无理函数分为显式的和隐式的。显式如上,而\textbf{隐式无理函数}是从方程中产生的,例如,
    \begin{equation*}
        Z^7 = azZ^2 - bz^5
    \end{equation*}
    
    而有理函数可以进一步被分为整函数和分数函数。整函数分母中不含$z$且变量$z$的指数为非负整数,形如,
    \begin{equation*}
        a + bz + cz^2 + dz^3 + ez^4 + fz^5 + \cdots
    \end{equation*}
    而分数函数形如,
    \begin{equation*}
        \frac{a + bz + cz^2 + dz^3 + ez^4 + fz^5 + \cdots}{\alpha + \beta z + \gamma z^2 + \delta z^3 + \epsilon z^4 + \zeta z^5 + \cdots}
    \end{equation*}
    
    还可以从另一个角度考虑函数的分类,即从变量$z$的每一个值能够得到一个确定的函数值。若只得到一个确定的函数值,则为单值函数。有理函数中的整函数和分数函数都是单值函数。而无理函数则是多值的,例如根号会给出两个值。超越函数有单值、多值和无穷多值的。例如弦函数有无穷多值。
    \begin{equation*}
        chord(\theta) = \sqrt{(1 - cos(\theta))^2 + sin^2(\theta)} = z \iff chord^{-1}(z) = \theta
    \end{equation*}
    
    用$P, Q, R, S, T$等表示$z$的单值函数。
    
    可以递归地用方程来定义$n$值函数。
    \begin{equation*}
        Z^n - PZ^{n-1} + QZ^{n-2} - RZ^{n-3} + SZ^{n-4} - \cdots = 0
    \end{equation*}
    这样$Z$就是$z$的$n$值函数。其对应的$n$个值中虚数的值必为偶数个。$n$值之和等于$P$,积等于$Q$,两个两个(组合)之积的和等于$Q$,以此类推。(如$3$值函数对应值$a, b, c$,$ab + ac + bc = Q$且$abc = R$)
    
    如果$z$的多值函数$Z$恒有且仅有一个实数值,那么$Z$可以被视作单值函数,如$p^{1/3}, p{1/5}$。形如$P^{\frac{m}{n}}$的函数,$n$为奇数则可当作单值,反之有0个或2个实根。($\frac{m}{n}$为最简形式)
    
    $y$是$z$的函数 $\iff$ $z$是$y$的函数。若$x$和$y$是$z$的函数,则若$x$和$y$互为函数。下面考虑一些特殊函数:
    \begin{itemize}
        \item 偶函数:$z$取$\pm k$时值相等的函数;偶函数又可分为单值偶函数和多值偶函数,即其多个值都相等。值得注意的是$z$的次数必为偶数,因此可定义偶函数为$z^2$的函数。
        \item 奇函数:把$z$变为$-z$后,函数值变号的即为奇函数;$z$的偶函数乘上$z$的奇函数,积为上$z$的奇函数;两个奇函数相乘或相除,得到偶函数;若$y$是$z$的奇函数,则$z$是$y$的奇函数;如果确定$y$为$z$的函数的方程中,各项中$y$和$z$的指数之和同偶或同奇,则$y$为$z$的奇函数。
        \item 相似函数:$Y$与$y$之间的函数关系同于$Z$与$z$之间的函数关系,则$Y$和$Z$是相似函数。
    \end{itemize}
    
\end{note}

% \begin{note}\label{circularity}
%     \emph{circularity}: using an advanced fact to prove a more elementary fact, and then later using the elementary fact to prove the advanced fact.
%     When do a mathematics proofs, one should avoid \emph{circularity}.
% \end{note}

% \begin{note}
%     From a logical point of view, there is no difference between a lemma, proposition, theorem, or corollary - they are all claims waiting to be proved.
%     However, we use these terms to suggest different levels of importance and difficulty.
%     A lemma is an easily proved claim which is helpful for proving other propositions and theorems, but is usually not particularly interesting in its own right.
%     A proposition is a statement which is interesting in its own right, while a theorem is a more important statement than a proposition which says something definitive on the subject, and often takes more effort to prove than a proposition or lemma.
%     A corollary is a quick consequence of a proposition or theorem that was proven recently.
% \end{note}