% 使用中文书籍文档类
\documentclass[12pt,oneside]{ctexbook}

%==============================================================================
% Preamble.
%==============================================================================

% Correctly showing characters outside ASCII.
\usepackage[T1]{fontenc}
% File is written and read with utf8 encoding.
\usepackage[utf8]{inputenc}
% Set paging layout.
\usepackage[margin=1.2in]{geometry}
% Including `amsfonts'.  Must be loaded before `mathtools'.
\usepackage{amssymb}
% Including `amsmath' and fixing bugs for `amsmath'.
\usepackage{mathtools}
% Must be loaded after `amsmath' and `mathtools'.
\usepackage{amsthm}
% Automatically adjust character spacing at margins.
\usepackage{microtype}
% Provide further utilities and fix bugs for `enumerate', `itemize' and
% `description'.
\usepackage{enumitem}
% Provide better quoting environment.
\usepackage{dirtytalk}
% Parsing list inside `\newcommand'.
\usepackage{listofitems}
% Nice looking if-then-else structure with comparison functionality.
\usepackage{ifthen}
% Automatically add hyperlinks to labels/refs.  Must be loaded after all
% packages above and before `cleveref'.  Recommend to use with `natbib' when
% you need bibtex.
\usepackage{hyperref}

\hypersetup{         % This macro come with `hyperref'.
	colorlinks=true, % Color hyperlinks.
	linkcolor=blue,  % Color local hyperlinks with blue.
	urlcolor=cyan,   % Color url links with cyan.
}

%------------------------------------------------------------------------------
% Define environments.
%------------------------------------------------------------------------------

% Text inside the body of theorem-like environments are set to Roman font.
% theorem-like environments share their counters, counters follow section and
% reset in every sections (except for axioms, axioms counters are reset in each
% chapter).  Exercises has their owned counter.  Notes do not use counter.
% See `amsthm' for details.
\theoremstyle{definition}
\newtheorem{axiom}{Axiom}[chapter]
\newtheorem{additional corollary}{Additional Corollary}[section]
\newtheorem{exercise}{Exercise}[section]
\newtheorem{theorem}{Theorem}[section]
\newtheorem{corollary}[theorem]{Corollary}
\newtheorem{definition}[theorem]{Definition}
\newtheorem{example}[theorem]{Example}
\newtheorem{lemma}[theorem]{Lemma}
\newtheorem{proposition}[theorem]{Proposition}
\newtheorem{remark}[theorem]{Remark}
\newtheorem*{note}{Note}

\theoremstyle{remark}
\newtheorem*{meta-proof}{Meta-proof}

% In `enumerate' enviroments, items' label are alphabets and surrounded by
% parentheses.  See `enumitem' for details.
\renewcommand{\labelenumi}{\textnormal{(}\alph{enumi}\textnormal{)}}

% Formatting exercises section.
\newcommand{\exercisesection}{
    \begin{center}
        --- Exercises ---
    \end{center}
}

%------------------------------------------------------------------------------
% Define operators and symbols.
%------------------------------------------------------------------------------

% Absolute value.
\DeclarePairedDelimiter\abs{\lvert}{\rvert}
% Ceiling.
\DeclarePairedDelimiter\ceil{\lceil}{\rceil}
% Floor.
\DeclarePairedDelimiter\floor{\lfloor}{\rfloor}

%==============================================================================
% Document.
%==============================================================================

\begin{document}

%------------------------------------------------------------------------------
% Front matters.
%------------------------------------------------------------------------------

\frontmatter

% Author informations.
\title{无穷分析引论笔记}
\author{Haocheng Xia}
\maketitle

% Table of contents.
\tableofcontents{}

%------------------------------------------------------------------------------
% Main matters.
%------------------------------------------------------------------------------

\mainmatter

% All chapters are in separated files.  We include them here.
\chapter{函数}\label{ch 1}

\begin{note}
    在分析中,a,b,c等代表常量,x,y,z等代表变量。确定的量(常量)都只可以是一个数,而变量包含每一个确定的量。指定变量为某个确定的值,它就成了常量。
    变量包容着正数和负数、整数和分数、有理数和超越数、零和虚数等一切数。
    
    变量的函数是变量、常量和数用某种方式联合在一起的解析表达式。只含一个变量$z$,其余均为常量的表达式为$z$的函数,
    \begin{equation*}
        a + 3z, az - 4z^2, az+b \sqrt{a^2 - z^2}, z^2
    \end{equation*}
    
    变量的函数本身也是一个变量。变量可以取虚数值,因为函数也可以去任何值。例如$\sqrt{9 - z^2}$。存在样子像函数的常量,
    \begin{equation*}
        z^0, 1^z, \frac{a^2 - az}{a - z}
    \end{equation*}
    
    函数之间的基本区别是变量和常量的联合方式,联合方式决定于运算。运算包括代数运算,即加、减、乘、除、乘方、开方和解方程;超越运算,即指数运算,对数运算以及积分学提供的运算。下面对包含多于一种运算的表达式加以分类。函数分为\textbf{代数函数}和\textbf{超越函数}。代数函数只含有代数运算,超越函数含有超越运算。代数函数常常不能显式表出。例如$z$的函数$Z$,
    \begin{equation*}
        Z^5 = az^2Z^3 - bz^4Z^2 + cz^3Z - 1
    \end{equation*}
    
    代数函数又分为有理函数和无理函数。有理函数的变量不受根号作用,无理函数反之。有理函数仅含加、减、乘、除和整数次乘方运算,
    \begin{equation*}
        a + z, a - z, az, \frac{a^2 + z^2}{a + z}, az^3 - bz^5
    \end{equation*}
    
    而无理函数形如,
    \begin{equation*}
        a + \sqrt{a^2 - z^2}, (a - 2z + z^2)^{\frac{1}{3}}, \frac{a^2 - z\sqrt{a^2 + z^2}}{a + z}
    \end{equation*}
    
    无理函数分为显式的和隐式的。显式如上,而\textbf{隐式无理函数}是从方程中产生的,例如,
    \begin{equation*}
        Z^7 = azZ^2 - bz^5
    \end{equation*}
    
    而有理函数可以进一步被分为整函数和分数函数。整函数分母中不含$z$且变量$z$的指数为非负整数,形如,
    \begin{equation*}
        a + bz + cz^2 + dz^3 + ez^4 + fz^5 + \cdots
    \end{equation*}
    而分数函数形如,
    \begin{equation*}
        \frac{a + bz + cz^2 + dz^3 + ez^4 + fz^5 + \cdots}{\alpha + \beta z + \gamma z^2 + \delta z^3 + \epsilon z^4 + \zeta z^5 + \cdots}
    \end{equation*}
    
    还可以从另一个角度考虑函数的分类,即从变量$z$的每一个值能够得到一个确定的函数值。若只得到一个确定的函数值,则为单值函数。有理函数中的整函数和分数函数都是单值函数。而无理函数则是多值的,例如根号会给出两个值。超越函数有单值、多值和无穷多值的。例如弦函数有无穷多值。
    \begin{equation*}
        chord(\theta) = \sqrt{(1 - cos(\theta))^2 + sin^2(\theta)} = z \iff chord^{-1}(z) = \theta
    \end{equation*}
    
    用$P, Q, R, S, T$等表示$z$的单值函数。
    
    可以递归地用方程来定义$n$值函数。
    \begin{equation*}
        Z^n - PZ^{n-1} + QZ^{n-2} - RZ^{n-3} + SZ^{n-4} - \cdots = 0
    \end{equation*}
    这样$Z$就是$z$的$n$值函数。其对应的$n$个值中虚数的值必为偶数个。$n$值之和等于$P$,积等于$Q$,两个两个(组合)之积的和等于$Q$,以此类推。(如$3$值函数对应值$a, b, c$,$ab + ac + bc = Q$且$abc = R$)
    
    如果$z$的多值函数$Z$恒有且仅有一个实数值,那么$Z$可以被视作单值函数,如$p^{1/3}, p{1/5}$。形如$P^{\frac{m}{n}}$的函数,$n$为奇数则可当作单值,反之有0个或2个实根。($\frac{m}{n}$为最简形式)
    
    $y$是$z$的函数 $\iff$ $z$是$y$的函数。若$x$和$y$是$z$的函数,则若$x$和$y$互为函数。下面考虑一些特殊函数:
    \begin{itemize}
        \item 偶函数:$z$取$\pm k$时值相等的函数;偶函数又可分为单值偶函数和多值偶函数,即其多个值都相等。值得注意的是$z$的次数必为偶数,因此可定义偶函数为$z^2$的函数。
        \item 奇函数:把$z$变为$-z$后,函数值变号的即为奇函数;$z$的偶函数乘上$z$的奇函数,积为上$z$的奇函数;两个奇函数相乘或相除,得到偶函数;若$y$是$z$的奇函数,则$z$是$y$的奇函数;如果确定$y$为$z$的函数的方程中,各项中$y$和$z$的指数之和同偶或同奇,则$y$为$z$的奇函数。
        \item 相似函数:$Y$与$y$之间的函数关系同于$Z$与$z$之间的函数关系,则$Y$和$Z$是相似函数。
    \end{itemize}
    
\end{note}

% \begin{note}\label{circularity}
%     \emph{circularity}: using an advanced fact to prove a more elementary fact, and then later using the elementary fact to prove the advanced fact.
%     When do a mathematics proofs, one should avoid \emph{circularity}.
% \end{note}

% \begin{note}
%     From a logical point of view, there is no difference between a lemma, proposition, theorem, or corollary - they are all claims waiting to be proved.
%     However, we use these terms to suggest different levels of importance and difficulty.
%     A lemma is an easily proved claim which is helpful for proving other propositions and theorems, but is usually not particularly interesting in its own right.
%     A proposition is a statement which is interesting in its own right, while a theorem is a more important statement than a proposition which says something definitive on the subject, and often takes more effort to prove than a proposition or lemma.
%     A corollary is a quick consequence of a proposition or theorem that was proven recently.
% \end{note}
\chapter{函数变换}\label{ch 2}
\begin{note}
改变函数形式的方法包括保持原有变量,改变表达式形式,
\begin{align*}
    2 - 3z + z^2 &\rightarrow (1-z)(2-z)\\
    \frac{1}{\sqrt{1 + z^2} - z} &\rightarrow \sqrt{1 + z^2} + z
\end{align*}
以及替换变量,也称换元,例如用$y$替换$a-z$,$$a^4 - 4a^3z + 6a^2z^2 - 4az^3 + z^4 \rightarrow y^4$$
本章仅考虑不换元的\textbf{改写表达式}替换。

往往可以将整函数分解成因式,对于关于变量$z$的函数,使用$z$的最高次数来区别因式。最高次为1的为线性因式,形如,
$$f + gz$$
而最高次为2的为二次因式,形如,
$$f + gz + hz^2$$
以此类推,n次因式是n个线性因式的积。

对于$z$的整函数$Z$,求出方程$Z=0$的所有根即为求出了所有线性因式。若$z=f$为根,即$z-f$除得尽$Z$,也即为因式。若$z = f,g,h,\cdots$为根,即$Z$的乘积形式为,
$$Z = (z-f)(z-g)(z-h)\cdots$$

线性因式可实可虚,实根给出实因式,虚根给出虚因式,根据\autoref{ch 1}若函数$Z$有虚因式,则其个数必为偶数。函数$Z$的全部虚因式之积必定为实因式,因为其等于$\frac{Z}{P}$,$P$为$Z$的全体实因式之积。

若Q是4个虚线性因式之积,则可表示为两个实二次因式之积。Q形如,
$$z^4 + Az^3 + Bz^2 + Cz + D$$

\begin{proof}
假定Q不能表示成两个实因式的乘积,那么一定可以表示成以下两个虚因式之积,
\begin{align*}
    z^2 &- 2(p + q \sqrt{-1})z + r + s\sqrt{-1}\\
    z^2 &- 2(p - q \sqrt{-1})z + r - s\sqrt{-1}
\end{align*}
由以上两个虚二次因式又可以得到以下4个虚线性因式,
\begin{align*}
    I&.z - (p + q \sqrt{-1}) + \sqrt{p^2 + 2pq\sqrt{-1} - q^2 - r - s\sqrt{-1}}\\
    II&.z - (p + q \sqrt{-1}) - \sqrt{p^2 + 2pq\sqrt{-1} - q^2 - r - s\sqrt{-1}}\\
    III&.z - (p - q \sqrt{-1}) + \sqrt{p^2 - 2pq\sqrt{-1} - q^2 - r + s\sqrt{-1}}\\
    IV&.z - (p - q \sqrt{-1}) - \sqrt{p^2 - 2pq\sqrt{-1} - q^2 - r + s\sqrt{-1}}
\end{align*}
而I、III之积,II、IV之积均为实。
\end{proof}

由于虚线性因式个数是偶数,而只有两个虚线性因式时,其乘积为实二次因式,上述证明又验证了4个虚线性因式时,其可表示为两个实二次因式,由此不妨得出猜想(此处未严格证明):$z$的整函数可以表示成实线性因式和实二次因式之积。

整函数的连续性,如果$z$的整函数$Z$在$z=a$时取值为$A$,在$z=b$时取值为$B$,那么对于一个在$A$和$B$之间的值$C$,一定存在一个$c$使得$z=c$时$Z=C$。因此若表达式$Z - A = 0$和$Z - B = 0$各有一个实线性因式,那么对$Z - C = 0$也有一个实线性因式。

如果最高次式是奇数,那么函数Z至少有一个实线性因式,可以从虚根为偶数个出发得证。另一种证明方法如下。
\begin{proof}
    $z = \infty$时,除了最高次的$z^{2n+1}$其他项可忽略,从而$Z - \infty$有因式$z - \infty$。$z = -\infty$时类似。因此若$Z=C$,且$C$在$-\infty$和$+\infty$之间时,必有实数$c$使得$Z - C = 0$。
\end{proof}
进一步可以得到最高次为奇数的情况,实线性因式的个数为奇数,因为若存在偶数个实线性因式的情况下,$Z$分离出这两个因式后最高次再次变为奇数,又至少有一个实线性因式。

而如果最高次为偶数,且常数A的符号为负,则$Z$至少有两个实线性因式,此时Z形如,
$$z^{2n} \pm \alpha z^{2n-1} \pm \beta z^{2n-2} \pm \cdots \pm \gamma z - A$$
\begin{proof}
    令$z=0$,$Z=-A$;令$Z=\pm \infty$,$Z = \infty$。又$-A < 0$,一定存在实根在$-\infty$到$0$上使得$Z=0$,在$0$到$\infty$同理。
\end{proof}

在分数函数中,若分子中$z$的最高次不小于分母的,则该函数(假分数函数)可表示为一个整函数和一个新的分数函数之和。新的分数函数中,分子最高次小于分母的。
$$\frac{1+z^4}{1+z^2} = z^2 - 1 + \frac{2}{1+z^2}$$

分母是两个互质因式(即这两个因式的因式互不相同)乘积的分数函数,可分解为分别以这两个因式作为分母的两个分数函数的和。例如,
$$\frac{1 - 2z + 3z^2 - 4z^3}{1 + 4z^4}$$
分母可分解为$1 + 2z + 2z^2$和$1 - 2z + 2z^2$。
假设有,
$$\frac{1+z^4}{1+z^2} = z^2 - 1 + \frac{2}{1+z^2} = \frac{\alpha + \beta z}{1 + 2z + 2z^2} + \frac{\gamma + \delta z}{1 - 2z + 2z^2}$$
可以得到,
\begin{gather*}
    \alpha + \gamma = 1\\
    -2\alpha + \beta + 2\gamma + \delta = -2\\
    2\alpha - 2\beta + 2\gamma + 2\delta = 3\\
    2\beta + 2\delta = -4
\end{gather*}
从而,
$$\frac{1+z^4}{1+z^2} = z^2 - 1 + \frac{2}{1+z^2} = \frac{\frac{1}{2} - \frac{5}{4} z}{1 + 2z + 2z^2} + \frac{\frac{1}{2} - \frac{4}{3} z}{1 - 2z + 2z^2}$$
由于引进的未知数个数恒等于分子的项数因此一定有解。

分数函数$\frac{M}{N}$可以被分解成$N$的不相同线性因式个数那么多个简分式,形如$\frac{A}{p - qz}$。例如,
$$\frac{1 + z^2}{z - z^3} = \frac{1}{z} + \frac{1}{1-z} - \frac{1}{1+z}$$
分母$N$的每一个线性因式都对应得到一个分数函数$\frac{M}{N}$分解式中的一个简分式,下面给出单个简分式的求法:
\begin{align*}
    \intertext{设$p-qz$是$N$的一个线性因式,即}
    N = (p - qz)S
    \intertext{此处S是z的整函数,把两个因式对应的分式记作$\frac{A}{p - qz}$和$\frac{P}{S}$,有}
    \frac{M}{N} = \frac{A}{p - qz} + \frac{P}{S} = \frac{M}{(p - qz)S}
    \intertext{即}
    \frac{P}{S} =  \frac{M - AS}{(p - qz)S}
    \intertext{可知$q - qz$是$M - AS$的因式。$z = \frac{p}{q}$时$A = \frac{M}{S}$,也就是说$A$等于$z = \frac{p}{q}$时,$\frac{M}{S}$的值。}
\end{align*}
应用实例,
$$\frac{1 + z^2}{z - z^3}$$
取线性因式$z$,那么此时$S = 1-z^2, M = 1 + z^2$,可得$A=1$。

\end{note}

%------------------------------------------------------------------------------
% Back matters.
%------------------------------------------------------------------------------

\backmatter

\end{document}